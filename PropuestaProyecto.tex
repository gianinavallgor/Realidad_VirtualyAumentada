% !TEX TS-program = pdflatex
% !TEX encoding = UTF-8 Unicode

% This is a simple template for a LaTeX document using the "article" class.
% See "book", "report", "letter" for other types of document.

\documentclass[11pt]{article} % use larger type; default would be 10pt

\usepackage[utf8]{inputenc} % set input encoding (not needed with XeLaTeX)
\usepackage[spanish]{babel}

%%% Examples of Article customizations
% These packages are optional, depending whether you want the features they provide.
% See the LaTeX Companion or other references for full information.

%%% PAGE DIMENSIONS
\usepackage{geometry} % to change the page dimensions
\geometry{a4paper} % or letterpaper (US) or a5paper or....
% \geometry{margin=2in} % for example, change the margins to 2 inches all round
% \geometry{landscape} % set up the page for landscape
%   read geometry.pdf for detailed page layout information

\usepackage{graphicx} % support the \includegraphics command and options

% \usepackage[parfill]{parskip} % Activate to begin paragraphs with an empty line rather than an indent

%%% PACKAGES
\usepackage{booktabs} % for much better looking tables
\usepackage{array} % for better arrays (eg matrices) in maths
%\usepackage{paralist} % very flexible & customisable lists (eg. enumerate/itemize, etc.)
\usepackage{verbatim} % adds environment for commenting out blocks of text & for better verbatim
\usepackage{subfig} % make it possible to include more than one captioned figure/table in a single float
% These packages are all incorporated in the memoir class to one degree or another...

%%% HEADERS & FOOTERS
\usepackage{fancyhdr} % This should be set AFTER setting up the page geometry
\pagestyle{fancy} % options: empty , plain , fancy
\renewcommand{\headrulewidth}{0pt} % customise the layout...
\lhead{}\chead{}\rhead{}
\lfoot{}\cfoot{\thepage}\rfoot{}

%%% SECTION TITLE APPEARANCE
\usepackage{sectsty}
\allsectionsfont{\sffamily\mdseries\upshape} % (See the fntguide.pdf for font help)
% (This matches ConTeXt defaults)

%%% ToC (table of contents) APPEARANCE
\usepackage[nottoc,notlof,notlot]{tocbibind} % Put the bibliography in the ToC
\usepackage[titles,subfigure]{tocloft} % Alter the style of the Table of Contents
\renewcommand{\cftsecfont}{\rmfamily\mdseries\upshape}
\renewcommand{\cftsecpagefont}{\rmfamily\mdseries\upshape} % No bold!

%%% END Article customizations

%%% The "real" document content comes below...

\title{Propuesta de Proyecto\\Libros con Realidad Aumentada}
\author{Roger Granda\\Gianina Vallejo}
%\date{} % Activate to display a given date or no date (if empty),
         % otherwise the current date is printed 

\begin{document}
\maketitle
\

\section{Introducción }

\paragraph{
Actualmente buscamos formas entretenidas de adquirir conocimiento.  Esto se evidencia en los diferentes institutos de instrucción, ya sea primaria, secundaria o universitaria; donde buscamos aprender sin tener ciertas limitantes o molestias, la información que encontremos tenga mucho texto que nos provoque sueño, al intentar mediante un libro explicarnos un tema que no sea tan comprensible al utilizar gráficos poco ilustrativos.
 } 


\paragraph{
La realidad aumentada es una tecnología que nos presenta videos, fotos, objetos, textos u otra forma virtual al mundo real que conocemos.  Esto nos abre un mundo inmenso de aplicaciones que le podemos dar.  Por ejemplo: aplicaciones publicitarias donde se muestra el producto sobre un patrón previamente reconocido, aplicaciones con información turística al enfocar la cámara del dispositivo móvil sobre un lugar reconocido de una ciudad, ect. .
 }

\paragraph{
La realidad aumentada también se puede aplicar en el ámbito educativo mejorando el aprendizaje de chicos y grandes.  
Nuestra aplicación consistirá en tener un libro con fotografías en una carilla y una pequeña descripción en la otra.  Al visualizar la fotografia mediante la cámara del dispositivo móvil se le incluirá artificialmente la opción de reproducir un video respecto a la imagen que esta observando o visualizar datos adicionales a esta.  Dependiendo la fotografia se mostrará una u otra opción.
 }

\section{Justificación}
\paragraph{
El motivo por el cuál se realiza este trabajo de investigación, es el de proveer una herramienta útil que sirva para demostrar las capacidades de la tecnología emergente de Realidad Aumentada.
 }
\section{Marco Teórico}
\paragraph{
De acuerdo a la deficinión de R.A., [1] señala que:\\*  } \begin{quote} La realidad aumentada es una tecnología que integra señales captadas del mundo real (típicamente video y audio) con señales generadas por computadores (objetos gráficos tridimensionales); las hace corresponder para construir nuevos mundos coherentes, complementados y enriquecidos – hace coexistir objetos del mundo real y objetos del mundo virtual en el ciberespacio-.\end{quote} 

\section{Análisis}
\paragraph{
El libro estará compuesto en su mayoría por ilustraciones, donde la realidad aumentada nos permitirá conocer más detalles sobre estas imágenes.  Sobre la imagen y mediante la cámara del dispositivo móvil se podrá observar en unos casos, recuadros con información adicional de la imagen o videos del tema central de la imagen.
 }
\paragraph{
Esto ayudará al aprendizaje de los más pequeños; donde encontrarán información de manera entretenida.  Así lograrán mayor comprensión de los temas esccolares.
 }
\paragraph{
Este proyecto se puede realizar con la ayuda del framework Vuforia; el cual nos provee mecanismos para reconocimiento de patrones.  Los patrones serían las imágenes del libro y sobre estas se proyectará información adiciconal.
 }

\section{Recursos}

\begin{enumerate} 
	 \item Dispositivo móvil
	 \item Cámara del dispositivo
	 \item Framework Vuforia
	 \item Libro con fotografias
 \end{enumerate}
\section{Bibliografía}
\begin{enumerate} 
 \item Revista Universitaria (2012). A REALIDAD AUMENTADA: UNA TECNOLOGÍA EN ESPERA DE USUARIOS. [ONLINE] Disponible en: http://www.revista.unam.mx/vol.8/num6/art48/jun\_art48.pdf. [Last Accessed 22/10/2013].

 \end{enumerate}

\end{document}
